%============================Acronimos e Nota��o =================================================

\chapter*{Lista de Acr�nimos e Nota��o}

\begin{tabular}{ll}
LMI  & Linear Matrix Inequality (desigualdade matricial linear)\\
LFT  & Linear Fractional Transformation (transforma��o linear fracion�ria)\\
LPV  & Linear Parameter-Varying (linear com par�metros variantes)\\
IQC  & Integral Quadratic Constraint (restri��o de integral quadr�tica)\\
\end{tabular}

\vspace*{1cm}

\begin{tabular}{ll}
$\star$ & indica bloco sim�trico nas LMIs\\
$L > 0$ & indica que a matriz $L$ � sim�trica definida positiva\\
$L \geq 0$ & indica que a matriz $L$ � sim�trica semi-definida positiva\\
$A$ & nota��o para matrizes (letras mai�sculas do alfabeto latino)\\
$A'$ & ($'$), p�s-posto a um vetor ou matriz, indica a opera��o de transposi��o\\
$\reais$ & conjunto dos n�meros reais\\
$\mathbb{Z}$ & conjunto dos n�meros inteiros\\
$\mathbb{Z}_+$ & conjunto dos n�meros inteiros n�o negativos\\
$\mathbb{N}$ & conjunto dos n�meros naturais (incluindo o zero)\\
$\I$ & matriz identidade de dimens�o apropriada\\
$\Z$ & matriz de zeros de dimens�o apropriada\\
$g!$ & s�mbolo (!), denota fatorial, isto �, $g!=g (g-1) \cdots (2) (1)$ para $g \in \mathbb{N}$\\
$N$ & especialmente utilizada para denotar o n�mero de v�rtices de um politopo\\
$n$ & especialmente utilizada para representar a ordem uma matriz quadrada\\
$\simplex$ & simplex unit�rio de $N$ vari�veis\\
$\alpha$ & especialmente utilizada para representar as incertezas de um sistema
\end{tabular}

\newpage
% verso em branco (sem numera��o na p�gina).


%==============================================================================================
