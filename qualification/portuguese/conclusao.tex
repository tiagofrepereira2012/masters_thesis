\chapter{Resultados e conclus�es parciais}

Foi avaliado 3 contramedidas bla bla bla.

Defini��o do Tau.

\section{Resultados preliminares}

\subsection{Protocolo de Avalia��o Intra Base de Dados}

A Tabela X exibe a performance obtida com cada contramedida utilizando o Protocolo de Avalia��o Intra Base de Dados.


\begin{table}[ht!]
\caption{$HTER(\%)$ of each countermeasure applying the intra-test ($D_1=D_2$) and the inter-test ($D_1 \neq D_2$) protocol.}
\begin{center}
  \begin{tabular}{ | c | c | c | c  c |  c  | }
    \hline

   \multirow{2}{*}{\textbf{Countermeasure}} & \textbf{Train/Tune} & \textbf{Test} & \multicolumn{2}{c|}{\textbf{HTER(\%)}} & \textbf{HTER degradation (test set) between}\\ 
     & EER in $D_1$ & $D_2$ & \textbf{dev} & \textbf{test}  & $D_1 = D_2$ and $D_1 \neq D_2$ \\ \hline
    
    \multirow{4}{*}{Correlation} & Replay  & Replay  &  11.66 & 11.79 & \multirow{2}{*}{$424.00\%$}\\ 
               &  \footnotesize{$EER=11.66\%$}& CASIA & 59.12 & 61.78  & \\ \cline{2-6}
               & CASIA  & Replay & 49.34 & 48.47  & \multirow{2}{*}{$54.56\%$}\\  
               & \footnotesize{$EER=24.91\%$}  & CASIA  & 24.91 & 31.36 & \\ \hline \hline

    \multirow{4}{*}{$LBPTOP_{8,8,8,1,1,1}^{u2}$}  & Replay & Replay  & 8.17 & 8.51  & \multirow{2}{*}{$499.88\%$} \\
               &  \footnotesize{$EER=8.17\%$}  & CASIA  & 52.32 & 51.05 &  \\ \cline{2-6}
               & CASIA  & Replay & 60.09 & 61.11  & \multirow{2}{*}{$174.40\%$} \\ 
               & \footnotesize{$EER=21.77\%$}   & CASIA  & 21.77 & 22.27 & \\ \hline \hline

    \multirow{4}{*}{$LBP_{8,1}^{u2}$} & Replay  & Replay  & 14.41 &15.45 & \multirow{2}{*}{$211.07\%$} \\
               &  \footnotesize{$EER=14.41\%$}  & CASIA  & 46.87  & 48.06  & \\ \cline{2-6}
               & CASIA  & Replay & 55.21 & 57.64  & \multirow{2}{*}{$155.72\%$} \\
               & \footnotesize{$EER=23.00\%$}  & CASIA  & 23.00  & 22.54 & \\
            
    \hline
  \end{tabular}
\end{center}
\label{tb:CrossTest}
\end{table}

Em termos de HTER � poss�vel observar um desempenho satisfat�rio de todas as contramedidas testadas em ambas as bases de dados. A an�lise da performance de cada contramedida no conjunto de calibra��o e no conjunto de teste sugere uma boa capacidade de generaliza��o em ambas as bases de dados j� que suas performances s�o semelhantes. Este comportamento pode ser tamb�m observado atrav�s das curvas ROC na Figura X. As curvas azuis e vermelhas (linha tracejada e linha s�lida) s�o as performances obtidas no conjunto de desenvolvimento e teste. � poss�vel observar que ambas as curvas est�o quase sobrepostas corroborando com o resultado obtido na tabela X.


\subsection{Protocolo de Avalia��o Inter Base de Dados}

A Tabela X exibe a performance obtida com cada contramedida utilizando o Protocolo de Avalia��o Inter Base de Dados.

Tabela.


Em termos de HTER � poss�vel observar um desempenho distante do obtido no experimento anterior sinalizando um forte enviesamento no processo de treinamento. Tal performance sugere que as contramedidas publicadas possui um poder de generaliza��o t�o bom quanto reportado. Este comportamento pode ser tamb�m observado atrav�s das curvas ROC na Figura X. As curvas azuis e verdes (linha tracejada e linha pontilhada) s�o as performances obtidas no conjunto de calibra��o de uma base de dados e no conjunto de teste de outra base de dados. � poss�vel observar que as curvas est�o bem distantes, ou seja, n�o � poss�vel ter uma performance satisfat�ria para cada valor de $\tal$. 



\section{Conclus�es parciais}


