\chapter{Metodologia}
\label{cap:metodology}

Neste cap�tulo ser�o apresentados as etapas de desenvolvimento deste projeto. Conforme

Primeiramente ser� .

Firstly, we study how the countermeasures, presented in Section \ref{sec:countermeasures}, will perform in a more realistic condition. This condition consists in training and tuning each one of the countermeasures with one face anti-spoofing database and testing with another one. To report the performance in such a scenario, two evaluation protocols were designed to work with the databases described in Section \ref{sec:Databases}. These protocols are the "intra-test" protocol and the "inter-test" protocol.


The intra-test protocol is equivalent to the database normal protocol. It consists in training, tuning and testing a countermeasure with the respectively training set, development set and test set of such a database. With this protocol, it is possible to evaluate the performance and the generalization power of a countermeasure within one database. The inter-test protocol evaluates the countermeasure performance in a more realistic scenario, close to real usage conditions. It consists in training and tuning a countermeasure with the training set and development set of one database and test it with the test set of another one. With this protocol, it is possible to evaluate the performance and the generalization power of a countermeasure in a set of unseen types of attacks.


Conforme descrito na se��o X, 




Escolha da base de dados (Replay e Casia)

Definir um protocolo de avalia��o HTER.

Como avaliar generaliza��o.