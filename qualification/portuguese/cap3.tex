\chapter{Metodologia}
\label{cap:metodology}

Neste cap�tulo ser�o apresentados as etapas de desenvolvimento deste projeto. Primeiramente � necess�rio avaliar se as contramedidas s�o realmente efetivas aplicando um protocolo �nico a fim de comparar diretamente cada contramedida. Como segunda contribui��o um protocolo para avalia��o do poder de generaliza��o foi proposto



\section{Protocolo de avalia��o Intra-base de dados}
\ref{sec:protocol}

Nesta avalia��o pretende-se avaliar a performance das contramedidas, em termos de detec��o de ataques, utilizando apenas uma base de dados.

Para compara��o escolhemos o Taxa m�dia de erro onde pondera HTER


\begin{equation}
\label{eq:HTER}
HTER=\frac{FAR(\tau,D)+ FRR(\tau,D)} {2} ,
\end{equation}
onde $\tau(D_n)$ � o limiar, $D_n$ � a base de dados, $FAR$ � a taxa de falsas aceita��es na base de dados $D_1$ e FRR � a taxa de falsas rejei��es na base de dados $D_1$.

\section{Protocolo de avalia��o Inter-base de dados}
\ref{sec:protocol}

Nesta avalia��o pretende-se avaliar a performance das contramedidas em termos de detec��o de ataques em um cen�rio mais real�stico. Este cen�rio consiste em treinar e calibrar as contramedidas utilizando uma base de dados e reportar os resultados utilizando outras bases de dados. 

\begin{equation}
\label{eq:HTER}
HTER(D_1)=\frac{FAR(\tau(D_2),D_1)+ FRR(\tau(D_2),D_1)} {2} ,
\end{equation}
onde $\tau(D_n)$ � o limiar, $D_n$ � a base de dados, $FAR$ � a taxa de falsas aceita��es na base de dados $D_1$ e FRR � a taxa de falsas rejei��es na base de dados $D_1$.