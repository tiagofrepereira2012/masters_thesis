%================================= Resumo e Abstract ========================================
\chapter*{Resumo}


\begin{quotation}
\noindent Autentica��o de usu�rios � uma tarefa crucial para proteger informa��es e nesta �rea a biometria de face apresenta algumas vantagens. A biometria de face � natural, f�cil de interagir e � uma das biometrias que possui um processo de coleta menos invasiva. Trabalhos recentes tem revelado que a biometria de face � vulner�vel a ataques de \textit{spoofing} utilizando equipamentos baratos. Contramedidas tem sido propostas para mitigar este tipo de vulnerabilidade. Por�m uma boa parte das contramedidas apresentadas na literatura s�o avaliadas utilizando m�tricas distintas e muitas vezes em bases de dados privadas impossibilitando uma compara��o honesta das mesmas. Este projeto tem como objetivo apresentar um estudo comparativo de contramedidas contra ataques de \textit{spoofing} em sistemas de autentica��o facial.


\vspace*{0.5cm}

\noindent Palavras-chave:  Antispoofing, Detec��o de vitalidade, Contramedidas, Reconhecimento Facial, Biometria
\end{quotation}


\newpage
\null



\chapter*{Abstract}


\begin{quotation}


\noindent User authentication is an important step to protect information and in this field face biometrics is advantageous. Face biometrics is natural, easy to use and less human-invasive. Unfortunately, recent work has revealed that face biometrics is vulnerable to spoofing attacks using low-tech equipments. Countermeasures have been proposed in order to mitigate this vulnerabilities. However several works in the literature present evaluations using different metrics and in private database making the comparison of countermeasures a difficult task. The main goal of this masters project is to provide a comparative study of countermeasures against \textit{spoofing} attacks.


\vspace*{0.5cm}

\noindent Key-words:  Antispoofing, Liveness detection, Countermeasure, Face Recognition, Biometrics
\newpage% verso em branco
\end{quotation}

\newpage
\null

