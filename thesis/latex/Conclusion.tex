\chapter{Conclusions}
\label{chap:Conclusions}

Focusing in antispoofing countermeasures for face authentication, the goal of this masters dissertation is two fold. The first one, we introduce a novel method to detect face spoofing using the spatiotemporal (dynamic texture) extensions of the Local Binary Pattern. The key idea of the approach is to learn and detect the structure and the dynamics of the facial micro-textures that characterise real faces but not fake ones. The second one, is to provide a comparative study of the state of the art countermeasures for face antispoofing. The key contribution of this comparative study is to covers tests in all video face antispoofing databases freely available focusing in the biases that these databases can introduce in the countermeasures.


As emphasized in the beginning, this masters dissertation focuses in a comparative study of antispoofing countermeasures for face authentication. 

The Chapter 

The intra-test protocol enabled us to measure the performance and evaluate the intra-database generalization of countermeasures. The evaluation of each countermeasure using this protocol, suggests that they are effective to detect spoofs in both databases. Even presenting different performances for different databases, the evaluated countermeasures presented a generalization capability. The only exception was the countermeasures based on eye blinks. With one eye blink, as a liveness check, this countermeasure was easy to deceive it, with a $FAR higher than 90\%$. Increasing the number of eye blinks the $FRR$ was higher than 90\%.

The inter-test protocol enabled us to evaluate the inter-database generalization of countermeasures. Using this protocol, was observed that the evaluated countermeasures accumulates a lot of bias from the databases. It was not possible to detect attacks from one database training the countermeasures with another database. It was observed two kinds of database bias. The first one, called \textbf{capture bias}, is a bias related to process of the databases construction. Both databases present different ways to carry out the attacks. The second one, called \textbf{attack bias}, is a bias related to the attacks. There are some attacks exclusive to the CASIA FASD and there are some exclusive to the Replay Attack Database.

In order to overcame these biases we introduce two approaches. The first one, combination of multiple databases, combines the train set of each database to train each one of the presented countermeasures. These strategy brought some improvement in performance, in HTER terms, compared with the both protocols, but the database bias still remains. In the second approach, we introduced the Score Level Fusion based Framework that merges the scores of countermeasures trained with different databases. The results obtained with the Score Level Fusion based Framework suggest that combining two good and not correlated countermeasures leads to significant improvement in performance, in HTER terms, compared with both protocols.


\section{Contributions}

This masters dissertation provided the following contributions:

\begin{enumerate}
	\item A reproducible research, releasing the all source code ....
	\item A promising countermeasure to detect spoof in a video sequences
	\item A method for compare  
\end{enumerate}

\section{Future work}



%As emphasized in the beginning, this master thesis was aimed at improving current face recognition performance rates by investigating the potential advantages of three different subjects related to this research field.

%As future work we will test more complex strategies of score fusion in order to improve the performance results. This Framework is flexible to aggregate not only data from different databases, but can support any kind of configuration. For example it is possible to aggregate countermeasures trained for a specific kind of attack (print attack, video attack, mobile phone attack and so on). Different configurations for the framework will be tested in the future.
