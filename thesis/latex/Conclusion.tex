\chapter{Conclusions}
\label{chap:Conclusions}

As emphasized in the beginning, the goal of this masters dissertation was two fold. Firstly, we introduced a novel method to detect face spoofing using dynamic textures. The key idea of the method was to analyse the structure and the dynamics of micro-textures in the facial regions using the $LBP-TOP$ texture descriptor. The $LBP-TOP$ provides an efficient representation for the countermeasure. The experiments carried out with this countermeasure consistently outperform prior works on the Replay Attack Database and in the CASIA FASD (following their provided protocols). Best results were achieved using nonlinear SVM classifier, but it is important to notice that experiments with simpler LDA based classification scheme resulted in comparable performance under various spoofing attack scenarios. The use of simple and computationally efficient classifiers should be considered when constructing real-world anti-spoofing solutions.

Secondly, we compared four countermeasures, representative according to the state of the art of this research field, using two different test protocols. Using the only two video face antispoofing databases publicly available (Replay Attack Database and CASIA FASD) we introduced the intra-test protocol and the inter-test protocol. The intra-test protocol enabled us to measure the performance and evaluate the intra-database generalization of countermeasures. The evaluation of each countermeasure using this protocol, suggests that they are effective to detect spoofs in both databases. Even presenting different performances for different databases, the evaluated countermeasures presented a generalization capability. The only exception was the countermeasures based on eye blinks. With one eye blink, as a liveness check, this countermeasure was easy to deceive it, with a $FAR$ higher than 90\%. Increasing the number of eye blinks the $FRR$ was higher than 90\%. The inter-test protocol enabled us to evaluate the inter-database generalization of countermeasures. Using this protocol, was observed that the evaluated countermeasures accumulates a lot of bias from the databases. It was not possible to detect attacks from one database training the countermeasures with another database. It was observed two kinds of database bias. The first one, called \textbf{capture bias}, is a bias related to process of the databases construction. Both databases present different ways to carry out the attacks. The second one, called \textbf{attack bias}, is a bias related to the attacks. There are some attacks exclusive to the CASIA FASD and there are some exclusive to the Replay Attack Database.

In order to overcame these biases we introduce two approaches. The first one, combination of multiple databases, combines the train set of each database to train each one of the presented countermeasures. This strategy brought improvements, in HTER terms, compared with the inter-test protocol, but was observed a strong bias to the Replay Attack Database. In the second approach, we introduced the Score Level Fusion based Framework that merges the scores of countermeasures trained with different databases. The results obtained with the Score Level Fusion based Framework suggest that combining two good and not correlated countermeasures leads to significant improvement in performance, in HTER terms, compared with both protocols. However, the literature lacks in video face spoofing databases; there are only two freely available. The effectiveness of the Score Level Fusion based Framework in a third video face spoofing database, at this stage is only speculative

\section{Contributions}

This masters dissertation provided the following contributions:

\begin{enumerate}
	\item A reproducible research. All source codes of this masters dissertation are freely available for download for future studies;
	\item An effective countermeasure against face spoofing attempts based on dynamic texture; 
	\item A comparative study on the state of the art countermeasures considering different databases and analysing possible biases that these databases can introduce in the countermeasures;
	\item A method inspired in an antivirus to combine different countermeasures.
\end{enumerate}


\section{Future work}

As future work, we can suggest:

\begin{enumerate}
	\item Explore different $LBP$ operators in the $LBP-TOP$ planes;
	\item Construction of new face antispoofing database in order to measure the effectiveness of the Score Level Fusion based Framework;
	\item Evaluate different fusion strategies in the Score Level Fusion based Framework;
	\item Evaluate different organizations for the Score Level Fusion based Framework. For example, it is possible to cover "micro" countermeasures, each one specialized in one type of attack. It is possible also to aggregate in the framework countermeasures that are complementary as in \cite{Komulainen_ICB_2013} or even consider the scores of a face verification as an element of the framework as in \cite{Chingovska_CVPRWORKSHOPONBIOMETRICS_2013}.

\end{enumerate}
