\chapter*{Abstract}

\begin{quotation}

\noindent 

User authentication is an important step to protect information and in this field face biometrics is advantageous. Face biometrics is natural, easy to use and less human-invasive. Unfortunately, recent work has revealed that face biometrics is vulnerable to spoofing attacks using low-tech equipments. The goal of this masters dissertation is two fold. Firstly, we introduce a novel and appealing approach to detect face spoofing using the spatiotemporal (dynamic texture) extensions of the highly popular local binary pattern operator. Evaluated with the only two publicly current available databases (Replay Attack Database and CASIA Face Anti-Spoofing Database), the final performance results show that our approach performs better than state of the art countermeasures following the provided evaluation protocols. Secondly, we provide a comparative study of countermeasures covering different databases and focusing in the biases that these databases can introduce. Evaluated with state of the art countermeasures, the results shown that the countermeasures accumulate a strong bias from the databases. 

\vspace*{0.5cm}

\noindent Key-words:  Antispoofing, Liveness detection, Countermeasure, Face Recognition, Biometrics
\newpage% verso em branco
\end{quotation}

\newpage
\null

