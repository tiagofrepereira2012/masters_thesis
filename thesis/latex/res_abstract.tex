\chapter*{Abstract}

\begin{quotation}

\noindent 

User authentication is an important step to protect information and in this field face biometrics is advantageous. Face biometrics is natural, easy to use and less human-invasive. Unfortunately, recent work has revealed that face biometrics is vulnerable to spoofing attacks using low-tech equipments. The goal of this masters dissertation is two fold. Firstly, we introduce a novel and appealing approach to detect face spoofing using the spatiotemporal (dynamic texture) extensions of the highly popular local binary pattern operator. Evaluated with the only two publicly current available databases (Replay Attack Database and CASIA Face Anti-Spoofing Database), the final performance results show that our approach performs better than state of the art countermeasures following the provided evaluation protocols. Secondly, we provide a comparative study of countermeasures covering different databases and focusing in the biases that these databases can introduce. Evaluated with state of the art countermeasures, the results shown that the countermeasures accumulate a strong bias from the databases. 

\vspace*{0.5cm}

\noindent Key-words:  Antispoofing, Liveness detection, Countermeasures, Face Recognition, Biometrics
\newpage% verso em branco
\end{quotation}

\newpage
\null



\chapter*{Resumo}

Autentica\c{c}\~ao de usu\'{a}rios \'{e} uma importante tarefa para proteger informa\c{c}\~{o}es e, nesta \'area de conhecimento, a biometria facial apresenta algumas vantagens. A biometria facial \'{e} natural, de usabilidade f\'{a}cil e menos invasiva. Infelizmente, trabalhos recentes revelaram que sistemas de autentica\c{c}\~{a}o facial s\~{a}o vulner\'{a}veis a ataques de \textit{spoofing} utilizando equipamentos baratos e de baixa tecnologia. Esta disserta\c{c}\~{a}o de mestrado possui dois objetivos principais. Primeiramente, apresentamos uma abordagem inovadora para detectar ataques de \textit{spoofing} em sistemas de autentica\c{c}\~{a}o facial utilizando texturas din\^{a}micas atrav\'{e}s de uma extens\~{a}o do descritor de textura \textit{Local Binary Patterns}. Experimentos realizados com as duas \'{u}nicas bases de dados de v\'{i}deo atualmente dispon\'{i}veis publicamente (\textit{Replay Attack Database} e \textit{CASIA Face Anti-Spoofing Database}), mostraram um desempenho superior \`{a}s contramedidas do estado da arte desta \'{a}rea de pesquisa. Como segundo objetivo, fornecemos um estudo comparativo de contramedidas cobrindo diferentes bases de dados, focando nos poss\'{i}veis vi\'{e}ses que estas bases de dados podem introduzir. Experimentos realizados com contramedidas do estado da arte, mostraram que as bases de dados introduzem um forte vi\'{e}s nas contramedidas.


\begin{quotation}
\noindent 

\vspace*{0.5cm}

\noindent Palavras-chave: Antispoofing, Detec\c{c}\~{a} de Vitalidade, contramedidas, Reconhecimento Facial, Biometria
\end{quotation}


\newpage
\null
